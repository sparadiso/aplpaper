\section{Qualification and Certification}
All of these opportunities are challenging for traditional methods of determining materials process and performance limits. How does a certification body create standards for processes with nearly infinite degrees of freedom? Perhaps additive manufacturing (AM), the industry equivalent of 3-d printing, is the best example of a leap forward in both capability and control in materials and product manufacture. Such advancements come with process tradeoffs and knowledge gaps\cite{Huang2015}. As such, the immediate impact of AM is being seen in tooling for reductive manufacturing approaches. As the industry matures, it will need to overcome the challenges inherent in creating degrees of control without a full understanding of how that control impacts long term and extreme performance\cite{Bikas2015}.

This more complicated data generated in the both the manufacurting and measurement of products using finely controlled tools demands a language with with to represent it in the form of data standards. While test certificates have existed for some time, even standardized data formats for electronic test certificates for engineering test data of materials have been identified as lacking and been called for\cite{Gagliardi2015357}. While the discussion around what these formats may look like continues, this call couples exactly with community coalescence around a similar data standards for fundamental materials data as well as processing information. By bringing the three of these together in compatible ways, both communication around data and advanced analysis enabled by data can be leveraged across the materials lifecycle. Private entities (Citrine Informatics), non-profit societies (ASME, ASTM), and government agencies (NIST) have proposed opens for open data frameworks, systems, and ontologies that flexibly accommodate data for purposes of recordkeeping and analysis\cite{CitrineInformatics,TheMineralsMetals&MaterialsSocietyTMS2015}. 

But there remains an additional level of complexity beyond the simple communication of data, and that is the use of dynamic materials system in performance critical structures and devices. Take, for example, the case of orientation controlled AM, which has been demonstrated recently\cite{DeHoff2015}. While control similar to this in microstructure, composition, physical dimension, and other processing undoubtedly can enable improved performance across a variety of metrics, but that control creates a level of analytical complexity that has not been seen before in manufacturing technologies creating a need for new approaches to materials certification that integrate materials data with process models and lifetime analyses. With increased pressure to shorten product development cycles, there is no doubt that the insertion of AM processes into production lines will feel similar pressures as designers and engineers begin to take advantage of AM specific capabilities\cite{Tyagi2015202}.
