\section{New Capabilities}

\subsection{Additive Manufacturing}

Additive Manufacturing (AM), the industry equivalent of 3-d printing, presents many new capabilities in the manufacturing process. These capabilities come with process tradeoffs and knowledge gaps. As such, the immediate impact of AM is being seen in tooling for reductive manufacturing approaches. As the industry matures, it will need to overcome the challenges inherent in creating degrees of control without a full understanding of how that control impacts long term and extreme performance\cite{Bikas2015}.

AM creates opportunities at every scale, but at each scale presents new challenges as well. First, take the macro-scale. Already, it has been demonstrated that objects as large a vehicle bodies can be created using AM processes\cite{Love_2015}. While this opens the door to more dynamic and customizable manufacturing lines with less waste, it upends auto industry norms around building commercial vehicles---frame designs, structural safety systems, quality assurance testing, among others.

Even in the face of changes at the architectural-scale, more impactful capabilities are being created at the product-scale. 

\begin{itemize}
\item AM creates lots of new capabilties
\item Human-scale -- parts of shapes and configurations never before possible can be built
\item Meso-scale -- graded compositions are possible; this creates big certifcation challenges
\item Micro-scale -- even microstructural grain control \cite{DeHoff2015


\item Data -- never before have we had particle-by-particle production data
\end{itemize}

\subsection{Qualification and Certification}
All of these opportunities are challenging for traditional methods of determining materials process and performance limits. How does a certification body create standards for these new capabilites?

Indeed, even standardized data formats for electronic test certificates for engineering test data of materials have been identified as lacking and been called for\cite{Gagliardi2015357}. While the discussion around what these formats may look like continues, this call couples exactly with community coalescence around a similar data standards for fundamental materials data as well as processing information. By bringing the three of these together in compatible ways, both communication around data and advanced analysis enabled by data can be leveraged across the materials lifecycle. Private entities (Citrine Informatics) and government agencies (NIST) have both proposed opens for open data frameworks that flexibly accommodate data for purposes of recordkeeping and analysis\cite{Citrination,TheMineralsMetals&MaterialsSocietyTMS2015}.

\begin{itemize}
\item model-based certification
\item data-based certification
\item continuous monitoring
\item lots of questions remain open -- this is a huge opportunity
\end{itemize}
