\section{Future Directions}
To date, materials informatics has in practice largely been focused on materials design and discovery on a laboratory scale. To truly leverage these discoveries going forward, the rest of the materials lifecycle must be considered holistically. This includes materials discovery, selection and optimization for product use, certification, and manufacturing. Materials informatics---including data standards, infrastructure, and analytics---creates and opportunity to link all of these stages as never before. The coupling of theory, data, and experiment will enable faster development at all stages, but many specific questions remain unanswered. As the field continues to mature, spurred on by government, non-profit, and private efforts, opportunities for new analyses, method development, and rapid empirical data collection will only grow. We expect that materials informatics as a field is already beginning to show results and will become a critical piece of the materials and product development process in the short-term.