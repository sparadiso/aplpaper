\section{Introduction}
The materials development process is sometimes viewed through a lens of materials discovery by researchers, processing and qualification by those that manufacture materials, and selection and lifecycle management by those that design products requiring advanced materials to be used within them. In their 2001 report, the National Research Council writes that ``new consumer product from invention to widespread adoption typically takes 2 to 5 years, but doing the same for a new material may take 15 to 20 years''\cite{NAP10187}. Important to note, is that this assertion does not include the process of actually inventing a new material at the lab scale. Indeed, even after the long and risky process of invention is over, there exists a further delay before a material can be effectively scaled from grams, to tons, to kilotons of production and finally integrated into production. In the strategic plan of the Materials Genome Initiative, connecting materials development, manufacturing, and lifecycle is identified as an important strategic goal\cite{MGIPlan}. 

This disconnect has been known for over 20 years. In 1995, NASA published guidelines around Technology Readiness Levels meant to systematize documentation of how close a new technology is to flight readiness\cite{TRL}. Shortly thereafter, the Department of Defense (DOD) identified that in addition to technology readiness, ability to manufacture the technologies---many of which are materials enabled---is critical to the deployment of those technologies. As such, the DOD created the Manufacturing Readiness Level system\cite{MRL}. These two systems emphasize that manufacturing and technology development must go hand-in-hand to achieve real-world impact.

While the high level problem is obvious and easily stated, the systems and norms required to accelerate the process are complicated because they cross disciplines, length scales, and corporate boundaries. They require working with advanced algorithms, cutting edge database technologies, and working with heterogeneous data. Addressed systematically, none of these is insurmountable; but similarly, none can be addressed in isolation. This paper will discuss the challenges and opportunities in the selection and manufacturing of advanced materials and the integration of materials informatics to enable better decisionmaking. In addition, we will address new capabilities in manufacturing and how informatics can enable the rapid deployment of these new manufacturing capabilities.