

\section{Manufacturing and Process Compatibility}

\textbf{REHASH WHY MFD IS IMPORTANT}\\
\\

\subsection{Challenges}
Bridging between these various 

\textbf{MARK JOHNSON STORY ABOUT ALUMINUM ALLOY THAT ISN'T MANUFACTURABLE AT SCALE}\\
\\
Historically, ICME has been focused on these relationships. A challenge remains, though, in the application of ICME-style approaches to materials systems for which deep mechanistic knowledge does not exist. 

Such problems are not limited to the world of structural materials. In the solid state lighting industry, making LED chips that hit very high target performance characteristics is relatively easy, but making a wafer of thousands of those chips remains an unsolved problem. The solid state lighting industry generally solves this by using binning to group similarly performing lamps into buckets, but this creates additional cost and may not be possible in all industries and at all scales. 

\subsection{Current Approaches}
In addition to the use of ICME in alloy development, the pharmaceutical community has made great strides in the connection of discovery, to testing, to manufacturing. There exist a great many simulation techniques model production fluid flows,

Even with a successful history of modeling materials in production, challenges loom on the horizon in parallel with what those of the materials industry: systems are becoming more complex while new, less well understood methods are being called for. This combination only increases the urgency for analytics systems to be integrated with manufacturing processes.\cite{JPS:JPS24594} Indeed, such issues have been well known in the pharmaceutical industry[citation?].


%Risk mitigation
%Mechanistic modeling
%Alloys
%ICME within the Alloys community has also enabled process modeling based on mechanistic understanding of processing-properties-%performance relationships. This approach enables the modeling of process variation to identify the performance characteristics of %manufactured alloys. \cite{MRS:10048345}
%Equipment monitoring

%Applicability across new classes of materials
%Complicated systems
%Lack of mechanistic knowledge

\subsection{Data Infrastructure and Sharing}
While pre-competitive knowledge sharing has been a norm in materials development, infrastructure for the sharing of raw data has only become broadly available over the last decade\cite{CitrineInformatics,MP}. 
While such resources enable the rapid evaluation of materials at the discovery and selection stages, a standard has not yet been established for sharing manufacturing process data because data at every stage of the materials development process is often considered trade secret\cite{TheMinerals2013}. 
Over the past decade, science has progressed to a more open culture around raw data sharing, but that culture has not yet led to any widespread data sharing in the manufacturing sector.
These challenges are substantial and will require cultural buy-in from multiple third parties in order to realize the efficiencies at a global scale that would result from open data exchange. 
Of course, the motivation for such a sea change in industrial thinking must be derived from confidence in the technical feasibility of the entire enterprise, and there are nontrivial technical hurdles related to storing representations of materials data that are sufficiently flexible to be useful across the industry that must be addressed. 

Most materials information, if structured at all, is stored in relational databases that impose a rigid schema on their data. 
For a problem of narrow scope, a relational structure can be feasible, but this architecture is fundamentally brittle: small changes to the original problem can precipitate large changes in the structure of the database or break the schema altogether and necessitate a new database for each use case.
This is a problem of high visibility in the field of information technology, and multiple solutions exist under the heading of so-called "NoSQL" databases that store data in machine parsable documents with no externally imposed schema\cite{Han2011,Kaur2013}. Materials scientists are already beginning to recognize the benefits of moving beyond a relational data model\cite{Blair2014}, and we believe that this is only the first of many coming realizations that the Big Data revolution has ushered in technological advancements for which the materials industry has demonstrable need.
For certain, the "materials genome" for describing metals, alloys, and polymers will all be dramatically different from one another, but conceptually there is no reason to prohibit comparing them on the basis of transferable properties such as yield strength, modulus, or toughness. 
In order to facilitate these comparisons, an open data platform with sufficient storage flexibility to hold all of the relevant data will need to be developed and embraced as a minimal first step, but the desired level of abstraction will be accomplished by embracing the extract, transform, load (ETL) model whereby a computation layer is deployed to translate disparate inputs into a common property format to facilitate comparison across material systems.

There are several government or nonprofit-led efforts to create testbed systems expressly for the purpose of enabling this degree of information exchange. These include ``The Smart Manufacturing Project'' run by the Smart Manufacturing Leadership Coalition and the recent Funding Opportunity Announcement for a Manufacturing Innovation Institute, both funded by the US Department of Energy\cite{SmartManufacturingLeadershipCoalition2013,EERE-MII}. These efforts, while in their early days, seek to create shared infrastructure, testbeds, and systems to motivate organizations of all types (e.g., universities, non-profit research labs, small businesses, and large corporations) to come together for common learning in precompetitive ways. 

One critical step to the implementation of testbeds is the creation of data standards. 

